\documentclass[
	%a4paper, % Use A4 paper size
	letterpaper, % Use US letter paper size
]{jdf}

\addbibresource{references.bib}
% Define custom command for parentheses citation
\newcommand{\pcite}[1]{(\cite{#1})}

\author{William Luna}
\email{wluna6@gatech.edu}
\title{Qualifier Question}

\begin{document}
%\lsstyle

\maketitle
%INSTRUCTIONS
%Before you can move on to proposing a contribution to the educational technology community, you need to demonstrate your mastery of the portion of the field to which you want to contribute. Because everyone’s interests and project ideas will be different, there is no question we can ask everyone that will be simultaneously deep enough to be challenging and general enough to cover everyone.

%So instead, on or by Friday of week 3, your mentor will send you a personal qualifier question targeted specifically to what you wrote on the first few assignments. This question will ask you to think deeply about the topics you have chosen. For the research track, it might ask you to synthesize and describe the viewpoints of different communities or research methodologies on your ideas. For the development track, it might ask you to describe the broader issues or pedagogical challenges associated with your intended designs. For the content track, it might ask you to consider elements of pedagogy and or instruction you have not yet considered.

%The primary goal of this assignment is to demonstrate your mastery of the portion of the field to which you want to contribute. In simpler words, the primary goal of this assignment is to show off what you know and how you can think. This is the closest thing this class has to a test. Adequately completing this assignment will require a significant command over the literature on your topic (in other words, several citations to others’ work in your area).

%Your assignment should be approximately 4 pages long in JDF. This is neither a minimum nor a maximum, but rather a heuristic to simply describe the level of depth we would like to see. Feel free to write more, or if you believe you can complete the assignment in fewer words, feel free to write less. Please make sure to include the question text itself so that your peer reviews know what you were asked to answer.

\section{Question}

\blockquote{Leveled readers have been a standard feature of K12 education for years and many best practices have developed over time.  What does the literature suggest are those principles/practices (eg. introducing more complex words and constructions in a controlled sequence and avoiding their use prior to that introduction)? How can an LLM best be induced to follow these practices?}

\section{Response}

- What are the best practices for leveled and/or graded readers
  - note the distinction between leveled and graded readers
    - note any relevant differences in pedagogy
      - students learning to read in their first language are learning to read words they already know, while ESL students are often encountering the vocabulary for the first time \cite{hu_2000}
  - introducing new words at a deliberate pace that facilitates vocabulary acquisition through context
    - based on Krashen's "i+1" theory that massive \textit{comprehensible} input is neccesary to learn new language material \cite{krashenreview}
    - comprehension begins to dip if the percent of known words in a text falls below ninety eight percent \cite{hu_2000}
    - uncommon words occur at a lower frequency in fiction that non-fiction, making fiction a better candidate for leveled readers \cite{hu_2000}
  - creating leveled readers rather than attempting to find content aimed at young native speakers
    - It requires knowing the 5000 most common words to know ninety eight percent of the words in young adult fiction \cite{nation1992vocabulary}
      - and students normally don't reach 5000 words until X amount of time studying (source)
    - it is difficult to simplify a text to use fewer than the 2,500 most common words of a language \cite{nation1992vocabulary}
    - as a corollary, graded reading schemes should seek to bring a student's vocabulary up to the 5000 most common words in the target language to enable them to engage with material geared towards native speakers (\cite{Nation2020GradedRA})
  - best retention of vocabulary from reading from "passive-first" environments where dictionaries are available but students are encouraged to first try and intuit or remember the meaning of unknown words \cite{mcdonald2016}.
    - providing fast lookup of words when necessary, since looking up words boosts retention, but discouraging excessive use of dictionary since it lowers reading speed \cite{dictionaryimportance}
    - Type of dictionary, including monolingual vs bilingual, has less impact on learning than how it is accessed \cite{dictionarypitfalls, imagesdictionary, bilingual_dictionary}.
    - Dictionaries are useful to all levels of foreign language student, but particularly helpful to beginners \cite{dictionaryvalue}
  - providing subsequent exposures to new words that follow the forgetting curve
    - in broadest terms and intuitively, the more times a new word is used in a text, the more likely the reader is to remember it \cite{clockworkorange}.
    - a word has to be encountered on average ten times before being cemented in long-term memory \cite{Nation2020GradedRA}.
    - enabled by spaced repetition software
      - FSRS emerging as best-performing algorithm \cite{shortestpathrepetitionscheduling}
    - have to consider lack of research that considers the possibility of inducing spaced repetition
      - since the texts are static and the intervals a student takes between reading sessions are dynamic
    - providing words in the text even after they have been committed to long-term memory \cite{cepedasrs}
  - introducing new grammar along a similar curve (?)
    - are there grammar dictionaries that can infer the grammatical structures in any selected sentence and explain them?
  - listening to audio transcriptions of text while reading it facilitates superior learning compared to reading the text on its own \cite{Chang2015ImprovingRR}
      
- The best way to induce an LLM to follow these practices
  - first, affirm that an LLM is the most effective way to simplify texts
    - LLMs outperform neural networks, machine learning frameworks, and other rules-based approaches
      - outperferm bespoke deep learning solutions (\cite{feng2023sentence}).
    - Open AI's models were the best during a recent evaluation \cite{kew2023bless} and GPT 4 stood out in particular \cite{wu2024indepth}
  - automating the simplification of existing stories to a lexical and grammatical level suitable for the student
    - evidence that students acquire vocabulary in a fluid, but generally predictable progression
    - evidence that LLMs one-shot task performance performs relatively well at this task
      - and evidence that context windows have become long enough to provide the full history of "known unknown" words
        - ie words that the student has self-identified as not being in long-term memory by virtue of looking up in a dictionary
        - despite evidence that LLMs have degraded retrieval of information provided in the middle of a context window, input tokens are sufficiently cheap and input window sufficiently large to provide details of several aspects of their vocabulary acquisition \cite{liu2023lost}
    - as already discussed, beginning with texts simplified to use the 2500 most common words in the target language strikes balance between comprehensibility and nuance \cite{nation1992vocabulary}.
  - use controllable story generation to adapt the story to the learning needs of the student 
    - track exposures of new words 
      - difficult to do perfectly since student has to actively self-identify which words they don't know in the passive activity of reading
        - and limited research
        - but possible through tracking which words a student looked up in a dictionary
          - which emphasizes the importance of an integrated dictionary
    - present "known unknown" words at a cadence that aligns with spaced repetition
      - Ebbinghaus' forgetting curve has been replicated several times in modern research settings \cite{Murre2015ReplicationAA}
    - track exposure of new grammar
      - under- explored, unclear how to surface grammatical challenges
  - establish an evaluation framework to model both the general and learner likelihood of comprehending the text (sources)
    - mention best current evals (sari), \cite{xu-etal-2016-optimizing}, (less used) SAMSA \cite{sulem-etal-2018-semantic}, BLEU, ROUGE, human, BLESS (more recent) \cite{kew2023bless}
      - pick only those that are used for LLM evals in future papers
      - eval framework particulary important since humans continue to outperform LLMs at sentencen simplification by a significant margin\cite{kew2023bless}
  - Use LLM to generate audio transcripts alongside the text it generates \cite{Chang2015ImprovingRR}
  - Tailwind effect of using LLM is that a dynamic text generation makes it easier to extend a graded reading system to 5000 words.
    - analysis of performance also makes it possible (though not necessarily requiring an LLM) to estimate the number of most common words the student knows.


\printbibliography{}

\end{document}