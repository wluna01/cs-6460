\documentclass[
	%a4paper, % Use A4 paper size
	letterpaper, % Use US letter paper size
]{jdf}

\addbibresource{references.bib}
% Define custom command for parentheses citation
\newcommand{\pcite}[1]{(\cite{#1})}

\author{William Luna}
\email{wluna6@gatech.edu}
\title{Qualifier Question}

\begin{document}
%\lsstyle

\maketitle
%INSTRUCTIONS
%Before you can move on to proposing a contribution to the educational technology community, you need to demonstrate your mastery of the portion of the field to which you want to contribute. Because everyone’s interests and project ideas will be different, there is no question we can ask everyone that will be simultaneously deep enough to be challenging and general enough to cover everyone.

%So instead, on or by Friday of week 3, your mentor will send you a personal qualifier question targeted specifically to what you wrote on the first few assignments. This question will ask you to think deeply about the topics you have chosen. For the research track, it might ask you to synthesize and describe the viewpoints of different communities or research methodologies on your ideas. For the development track, it might ask you to describe the broader issues or pedagogical challenges associated with your intended designs. For the content track, it might ask you to consider elements of pedagogy and or instruction you have not yet considered.

%The primary goal of this assignment is to demonstrate your mastery of the portion of the field to which you want to contribute. In simpler words, the primary goal of this assignment is to show off what you know and how you can think. This is the closest thing this class has to a test. Adequately completing this assignment will require a significant command over the literature on your topic (in other words, several citations to others’ work in your area).

%Your assignment should be approximately 4 pages long in JDF. This is neither a minimum nor a maximum, but rather a heuristic to simply describe the level of depth we would like to see. Feel free to write more, or if you believe you can complete the assignment in fewer words, feel free to write less. Please make sure to include the question text itself so that your peer reviews know what you were asked to answer.

\section{Question}

\blockquote{Leveled\footnote{The response will consider leveled and graded readers to be interchangeable terms, but noting that literature occasionally cleaves a distinction between the former's tendency to refer to readers that facilitate primary school students learning to read in their native language, and the latter's tendency to facilitate learning a foreign language through reading. Since there is evidence to suggest that learning to read in one's native language involved mostly learning to read already-known words \pcite{hu_2000}, as opposed to encountering a larger percentage of unknown words when reading in a foreign language, the distinction can have pedagogical implications, although they have limited bearing on this response. This statement will focus on the \textit{latter} use case, creating reading material for students of a foreign language.} readers have been a standard feature of K12 education for years and many best practices have developed over time.  What does the literature suggest are those principles/practices (eg. introducing more complex words and constructions in a controlled sequence and avoiding their use prior to that introduction)? How can an LLM best be induced to follow these practices?}

\section{Response}

\subsection{Leveled readers have been a standard feature of K12 education for years and many best practices have developed over time.  What does the literature suggest are those principles and/or practices?}

While there has been considerable evolution in the learning science underpinning the use of leveled readers, few would argue the impact of Stephen Krashen's Applied Linguistics research. In particular, Krashen's theory of Massive Comprehensible Input laid a foundational argument that language is acquired most effectively by asking students to reach just beyond their current semantic and grammatical ability when ingesting written and spoken language, sometimes referred to as "i+1" \pcite{krashen1982principles, krashenreview}. A more direct (and less academic) version of this takeaway is the perhaps intuitive observation that time spent free reading correlates to overall literacy \pcite{krashen_2004}. A few years before Krashen's seminal paper, Lev Vygotsky coined the term Zone of Proximal Development, in many ways representing the culmination of his decades of research into child development. ZPD is too general a concept to be considered an analogy for Krashen's more specific theories of language acquisition, but is certainly an influential precursor \pcite{vygotsky, vygostky_krashen_not_same}.

However, these theories stop short of providing an objective measure of comprehensible. How much novel vocabulary and grammar can a text present to a student without sacrificing their reading comprehension? Evidence shows that comprehension begins to dip if the percent of known words in a text falls below ninety-eight percent \pcite{hu_2000}. Unfortunately, this is a tall order for beginning and intermediate students. Ninety-eight percent comprehension of the words in young adult fiction requires knowing the 5,000 most common words in English \pcite{nation1992vocabulary}, which the Foreign Service Institute estimates requires a minimum of six hundred hours of study \pcite{fsi_language_learning}. Given that most American public high schools have one-hundred eighty days of instruction in a year, a student who begins a foreign language their freshman year, studying an hour each school day for all four years of high school, would fail to reach this level by the time they graduate.

The above hypothetical demonstrates the need for leveled readers, but what are the best practices for creating them? For one, fiction is a superior candidate for leveled readers, since uncommon words occur at a lower frequency in fiction that non-fiction \pcite{hu_2000}. Since students read for longer periods of time when interested in the subject matter, it's important to provide a variety of reading material that caters to each student's interests. \pcite{llm_augmented_exercise_retrieval}. Research has also shown that simplified reading material can lose too much nuance when simplified below the 2,500 most common words of a language, implying that leveled readers are difficult to introduce until a student has reached an advanced beginner proficiency in the foreign language \pcite{nation1992vocabulary}. 

Besides the general notion of producing a work of fiction-or simplifying an existing one–to a student's current understanding of syntax, semantics, and lexicon, is it possible to identify more specific guiding best practices? One such recommendation is to present specific vocabulary in the text enough times to commit it into memory. The fact that the more an individual encounters a word the more likely they are to remember it has been proven empirically \pcite{clockworkorange}, where a word has to be encountered an average of ten times before being cemented into long-term memory \pcite{Nation2020GradedRA} with continuous exposure to avoid forgetting it \pcite{cepedasrs}.

Independently of frequency, extensive research has been done into the optimal interval of time between presenting repetition of a word being studied by a language learner. Ebbinghaus' Forgetting Curve has been replicated several times since it was initially proposed in the late 1800s \pcite{Murre2015ReplicationAA}. It proposes that human memory mimics a logarithmic function, where an entirely novel piece of information is quickly forgotten unless promptly recalled, and that each more time can pass between each subsequent recall of that information without forgetting it. This finding implies that a leveled reader would be most effective if it repeated newly learned words in its text on deliberate intervals that followed the Forgetting Curve \pcite{shortestpathrepetitionscheduling}.

Dictionary use is another critical topic when discussing leveled readers. Should a student reference a dictionary at all when engaged in free extensive reading? If so, how? What is the ideal format of that dictionary? One study suggests that vocabulary retention is greatest in a "passive-first" environment, where dictionaries are available but students are encouraged to first try and intuit or remember the meaning of unknown words \pcite{mcdonald2016}. Another study drew a more moderate conclusion, recommending a balance between looking up words in a dictionary (to boost retention) but discouraging its excessive (to avoid lower reading speeds) \pcite{dictionaryimportance}. Dictionaries have been found to be particularly helpful to beginners but remain useful to all levels of foreign language student \pcite{dictionaryvalue}. And additional research suggests that whether a dictionary is monolingual, bilingual, or image based has less impact on learning outcomes than how frequently it is referenced \pcite{dictionarypitfalls, imagesdictionary, bilingual_dictionary}.

Finally, given that technology has a come a long way since leveled readers were first introduced, it should come as no surprise that some innovative modifications have been proposed. There is general consensus that listening to the audio transcription of a text while simultaneously reading it facilitates superior learning compared to reading the text on its own \pcite{Chang2015ImprovingRR}. While novel computer-assisted presentations of graded text such as interlinear reading and the diglot weave have been introduced in consumer language learning applications, research into their efficacy is inconclusive \pcite{hyplern_interlinear_reading, diglot_weave}.

\subsection{How can an LLM best be induced to follow these practices?}
      
- The best way to induce an LLM to follow these practices
  - first, affirm that an LLM is the most effective way to simplify texts
    - LLMs outperform neural networks, machine learning frameworks, and other rules-based approaches
      - outperferm bespoke deep learning solutions (\cite{feng2023sentence}).
    - Open AI's models were the best during a recent evaluation \cite{kew2023bless} and GPT 4 stood out in particular \cite{wu2024indepth}
    - Humans show a preference for the output of a Fine-tuned general purpose model over the same model without fine-tuning or a smaller purpose-build model \cite{ai_human_taking_turns_creating_story}
  - automating the simplification of existing stories to a lexical and grammatical level suitable for the student
    - evidence that students acquire vocabulary in a fluid, but generally predictable progression
    - evidence that LLMs one-shot task performance performs relatively well at this task
      - and evidence that context windows have become long enough to provide the full history of "known unknown" words
        - ie words that the student has self-identified as not being in long-term memory by virtue of looking up in a dictionary
        - despite evidence that LLMs have degraded retrieval of information provided in the middle of a context window, input tokens are sufficiently cheap and input window sufficiently large to provide details of several aspects of their vocabulary acquisition \cite{liu2023lost}
    - as already discussed, beginning with texts simplified to use the 2500 most common words in the target language strikes balance between comprehensibility and nuance \cite{nation1992vocabulary}.
  - use adaptive learning to adapt the story to the learning needs of the student 
    - adaptive learning models student's performance on past ecercises to create new exercises (or in this case, text) at an "appropriate estimated level of difficulty for that particular student." \cite{important_adaptive_learning_exercise_generation}
    - track exposures of new words 
      - difficult to do perfectly since student has to actively self-identify which words they don't know in the passive activity of reading
        - and limited research
        - but possible through tracking which words a student looked up in a dictionary
          - which emphasizes the importance of an integrated dictionary
    - present "known unknown" words at a cadence that aligns with spaced repetition
      - Ebbinghaus' forgetting curve has been replicated several times in modern research settings \cite{Murre2015ReplicationAA}
    - track exposure of new grammar
      - under- explored, unclear how to surface grammatical challenges
    - mimic the overall arc and tone of existing stories, ie controllable story generation \cite{controllable_story_generation}
  - Raises question of how to linking the stochastic output from an LLM into a traditional software system (SRS)
    - poses unique software design challenges but is possible \cite{taiwan_adaptive_testing}
    - alternatively can use deep knowledge tracing \cite{deep_knowledge_tracing} that generates content based on a model of that student's knowledge
      - although few knowledge-tracing models exist of the shelf, requiring the added complexity of deep learning to implement \cite{question_generation_adaptive_education, generative_information_retrieval}
        - and evidence that such a system improves upon FSRS is limited \cite{flashcard_scheduler_evolution}
        - and having the entire system rely on deep learning makes it difficult to interpret, making it harder to be used by a teacher \cite{deep_learning_knowledge_tracing}
        - and knowledge tracing models are most effective when trained for a specific domain \cite{dkt_knowledge_tracing}
      - knowledge tracing can be paired with a model of students' interests to strike a balance between what they need to practice and what they want to practice, although this its non-trivial to implement such a system \cite{llm_augmented_exercise_retrieval}
  - establish an evaluation framework to model both the general and learner likelihood of comprehending the text (sources)
    - mention best current evals (sari), \cite{xu-etal-2016-optimizing}, (less used) SAMSA \cite{sulem-etal-2018-semantic}, BLEU, ROUGE, human, BLESS (more recent) \cite{kew2023bless}
      - pick only those that are used for LLM evals in future papers
      - eval framework particulary important since humans continue to outperform LLMs at sentencen simplification by a significant margin\cite{kew2023bless}
  - Use LLM to generate audio transcripts alongside the text it generates \cite{Chang2015ImprovingRR}
  - Tailwind effect of using LLM is that a dynamic text generation makes it easier to extend a graded reading system to 5000 words.
    - analysis of performance also makes it possible (though not necessarily requiring an LLM) to estimate the number of most common words the student knows.
  - limitations
    - non-deteriminstic nature of LLM output. How to address expletives, plagiarism, or nonsense statements if they arise? \cite{recent_story_generation_review}


\printbibliography{}

\end{document}