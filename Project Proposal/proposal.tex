\documentclass[
	%a4paper, % Use A4 paper size
	letterpaper, % Use US letter paper size
]{jdf}

\addbibresource{references.bib}

\author{William Luna}
\email{wluna6@gatech.edu}
\title{Project Propsal}

\begin{document}
%\lsstyle

\maketitle
\section{Header}
%Every proposal should start with a title and a list of team members.
\subsection{Team Title: Enabling Spaced Repetition in Graded Readers}
\subsection{Team Members: Will Luna}

\section{Introduction}
%Briefly introduce your topic. Your goal in the introduction is to set up a narrative for why your project is actually valuable.
%If you’re on the Development track, this would frame the problem, specifically referencing why it’s a problem and why existing solutions aren’t sufficient, in order to leave room for you to contribute something meaningful.

Flashcards are a ubiquitous fixture in education. Most students are familiar with drilling multiplication tables, cramming unfamiliar words for a vocabulary test, or any other act of rote memorization facilitated by flash cards.

And–at least when learning our native language, and perhaps a foreign one–extensive reading is equally unavoidable. As students transition from "learning to read" to "reading to learn", they enter a phase of incidentally acquiring vocabulary encountered in text.

These two study methods lie on opposite ends of a spectrum. Consider the student who has to learn the word "muggle" for an upcoming vocabulary quiz. One study option is to read enough chapters of Harry Potter to encounter the word enough times to infer its definition in context. This might take hours if the student has to read several chapters before the word appears enough to cement it into their memory. Another option is to review a flashcard that contains the word muggle on one side and its definition on the other. The student can perform several reviews each minute.

Few would disagree that reading Harry Potter is more fun than studying flashcards. Reading has the added benefit of exposing the student to new vocabulary in addition to muggle, or entrenching understanding of recently encountered words. But if the goal is simply to memorize the word muggle, rote memorization via flashcards is certainly faster.

So where extensive reading optimizes for engagement, flashcards optimize for efficiency. 

But what if, through procedurally generating the text of reading material, a student could take advantage of the optimized scheduling of flashcard review while free reading?

Independent of this observation is the rapid advancement of the technology that might enable it. While even a few years ago it would have been impossible to conceive of such a system, today's SOTA (state-of-the-art) LLMs make it feasible to attempt. An LLM can receive a student's current language level, study schedule, and reading material as input, and generate as output a continuation of that reading material, at an appropriate difficulty level and interspersed with the vocabulary in need of memorization.

This hypothesis is the bedrock of my project proposal. That a hybrid approach would encourage students to study more and learn faster than if they were to engage in free reading and flashcard study separately.

\section{Related Work}
% Cover what others in the same area have done. This sets up the foundation for your work and tells the reader how what you will do is different from what is already out there, as well as how they should interpret the results of what they do in a broader context.
%If you’re on the Development track, this may mean other tools targeting the same area, although you’ll absolutely want to cover tools developed in both industry and academia. Academic tools tend to have more robust results published about their successes and failures.

Much work has already been done on both flashcard optimization and extensive reading as \textit{independent} research pursuits.

While flashcards have potentially been in use for thousands of years, many agree that the science of how to optimize their use began with Ebbinghaus' work on the Forgetting Curve in the 1880s. His observation, that information sticks in the human mind longer with each subsequent recall, has been replicated in multiple modern laboratory settings \cite{Murre2015ReplicationAA}. This finding became the bedrock of the SuperMemo method, an rule set that optimizes time spent studying flashcards by presenting harder to remember information to the student more often (and vice versa), and increasing the intervals between studying the same card each time a student correctly recalls it \cite{SuperMemo_Method_2023}. Supermemo remained the de-facto operating system of most flashcard software until 2022, when a Deep Learning model of student memory known as FSRS (Free Spaced Repetition Scheduler) was proposed \cite{shortestpathrepetitionscheduling}. It is currently the default scheduler powering Anki, the world's most popular Flashcard software.

Extensive reading pedagogy also continues to benefit from academic research.

\section{Proposed Work}
% This is the crux of the proposal. What are you going to do? Your description of your proposed work should be detailed enough that you could hand this proposal to someone else and they may be able to implement it themselves. We would expect every proposal to have subsections to the proposed work, but what those subsections are will differ based on your project.
%If you’re on the Development track, this would describe the tool, including what it will look like, how the user will access it, what languages or libraries it will be build in, etc. You might also have a section on evaluating the tool. If you are working on a team, you should note throughout the proposed work who will be responsible for what general parts. You will go into more detail on this in the task list. You may also want to include fall-back plans for portions of the work that may be unpredictable: for example, what will you do if you cannot recruit enough participants for a study, or if you are unable to integrate a certain pair of tools?


\section{Deliverables}
%As part of the project, you will produce two intermediate milestones, as well as the final project. Describe what these deliverables will be. Take a look at the course calendar to see when the intermediate deliverables are expected.
%Second, describe what you expect to be in your final project deliverable. This could be code, data, artifacts, courseware, videos, etc.



\section{Task List}
%At the conclusion of the proposal PDF, your proposal must have a task list. To create your task list:
%Download or make a copy of the task listLinks to an external site. template. Delete the sample tasks.
%Fill out the task list. You may add or remove rows as necessary. If you have multiple team members, you’ll definitely need to add rows.
%https://docs.google.com/spreadsheets/d/1T90bOeFfxmQAO-l6lA_L-XOaZyUWOibaZp0GWcyk0c4/edit#gid=0
%Make sure to set aside some time for preparing each milestone, the final paper, and the final presentation.
%Copy your task list into your proposal document.
%We recommend using the task list to monitor your progress throughout the semester. Each week, you should re-outline the rest of the semester to ensure you and your mentor remain in sync about your progress and expected final project.



\printbibliography{}

\end{document}

