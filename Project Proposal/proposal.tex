\documentclass[
	%a4paper, % Use A4 paper size
	letterpaper, % Use US letter paper size
]{jdf}

\addbibresource{references.bib}

\author{William Luna}
\email{wluna6@gatech.edu}
\title{Project Propsal}

\begin{document}
%\lsstyle

\maketitle
\section{Header}
%Every proposal should start with a title and a list of team members.


\section{Introduction}
%Briefly introduce your topic. Your goal in the introduction is to set up a narrative for why your project is actually valuable.
%If you’re on the Development track, this would frame the problem, specifically referencing why it’s a problem and why existing solutions aren’t sufficient, in order to leave room for you to contribute something meaningful.


\section{Related Work}
% Cover what others in the same area have done. This sets up the foundation for your work and tells the reader how what you will do is different from what is already out there, as well as how they should interpret the results of what they do in a broader context.
%If you’re on the Development track, this may mean other tools targeting the same area, although you’ll absolutely want to cover tools developed in both industry and academia. Academic tools tend to have more robust results published about their successes and failures.


\section{Proposed Work}
% This is the crux of the proposal. What are you going to do? Your description of your proposed work should be detailed enough that you could hand this proposal to someone else and they may be able to implement it themselves. We would expect every proposal to have subsections to the proposed work, but what those subsections are will differ based on your project.
%If you’re on the Development track, this would describe the tool, including what it will look like, how the user will access it, what languages or libraries it will be build in, etc. You might also have a section on evaluating the tool. If you are working on a team, you should note throughout the proposed work who will be responsible for what general parts. You will go into more detail on this in the task list. You may also want to include fall-back plans for portions of the work that may be unpredictable: for example, what will you do if you cannot recruit enough participants for a study, or if you are unable to integrate a certain pair of tools?


\section{Deliverables}
%As part of the project, you will produce two intermediate milestones, as well as the final project. Describe what these deliverables will be. Take a look at the course calendar to see when the intermediate deliverables are expected.
%Second, describe what you expect to be in your final project deliverable. This could be code, data, artifacts, courseware, videos, etc.



\section{Task List}
%At the conclusion of the proposal PDF, your proposal must have a task list. To create your task list:
%Download or make a copy of the task listLinks to an external site. template. Delete the sample tasks.
%Fill out the task list. You may add or remove rows as necessary. If you have multiple team members, you’ll definitely need to add rows.
%https://docs.google.com/spreadsheets/d/1T90bOeFfxmQAO-l6lA_L-XOaZyUWOibaZp0GWcyk0c4/edit#gid=0
%Make sure to set aside some time for preparing each milestone, the final paper, and the final presentation.
%Copy your task list into your proposal document.
%We recommend using the task list to monitor your progress throughout the semester. Each week, you should re-outline the rest of the semester to ensure you and your mentor remain in sync about your progress and expected final project.



\printbibliography{}

\end{document}

