\documentclass[
	%a4paper, % Use A4 paper size
	letterpaper, % Use US letter paper size
]{jdf}
\addbibresource{../../references.bib}
\author{William Luna}
\email{wluna6@gatech.edu}
\title{Assignment 1: Research Log}

\begin{document}
%\lsstyle

\maketitle

\section{Background}
% In about half a page, summarize your current state. This would largely cover where you left off last week.
Picking up from last week's \textbf{Planning} section, I felt that my literature review validated the desire to leverage an LLM to create graded reading content for second language learners. Those papers provided evidence that there is a need for such a tool, that building one seems technically feasible, and that there are multiple potential principles in learning science to leverage.

My goal for this week is hone my research into 

\begin{enumerate}
    \item 
\end{enumerate}

\section{Papers}
%As you walk through the Research Guide, you’ll be finding lots of papers to read. Here, you’ll make a list of the papers you come across and give considerable attention to. We would expect the Research Log to include at least 15-20 sources (though more is fine as well), and at least 12 (preferably more) should be academic and peer-reviewed. You may include blog posts, newspaper articles, etc. as well, but you should have at least 12 academic sources, too.

\subsection{\fullcite{caines2023application}}
%The paper’s bibliographic information (its APA citation, typically)
\subsubsection{Identification}
%In around one sentence, how you found it (a Google Scholar search? From a conference’s proceedings? From another paper’s references? %Something else?)
Prompted Chat GPT to provide papers on dynamic text generation for the purpose of foreign language learning.

\subsubsection{Summary}
%In around three sentences, a brief, original summary in your own words

\subsubsection{Takeaways}
%In around three sentences, the main takeaways going forward

\subsection{\fullcite{pilán2016readable}}
%The paper’s bibliographic information (its APA citation, typically)
\subsubsection{Identification}
%In around one sentence, how you found it (a Google Scholar search? From a conference’s proceedings? From another paper’s references? %Something else?)
This paper was provided from the same Chat GPT search as the one above.

\subsubsection{Summary}
%In around three sentences, a brief, original summary in your own words

\subsubsection{Takeaways}
%In around three sentences, the main takeaways going forward

\subsection{\fullcite{liu2023lost}}
%The paper’s bibliographic information (its APA citation, typically)
\subsubsection{Identification}
%In around one sentence, how you found it (a Google Scholar search? From a conference’s proceedings? From another paper’s references? %Something else?)
Found this paper searching google for papers focused on examining how many words map onto how many tokens of a context window, and to what extent LLM retrieval degrades as a function of longer context windows.

\subsubsection{Summary}
%In around three sentences, a brief, original summary in your own words

\subsubsection{Takeaways}
%In around three sentences, the main takeaways going forward

\subsection{\fullcite{mlq2023tokens}}
%The paper’s bibliographic information (its APA citation, typically)
\subsubsection{Identification}
%In around one sentence, how you found it (a Google Scholar search? From a conference’s proceedings? From another paper’s references? %Something else?)
Found from the same Google search. Not a research paper but seemed very informative.

\subsubsection{Summary}
%In around three sentences, a brief, original summary in your own words

\subsubsection{Takeaways}
%In around three sentences, the main takeaways going forward
Take away is that I don't need to be that worried about context windows since any given graded reader shouldn't be more than 5,000 words, at least for the sake of this exercise.


\subsection{\fullcite{cepedasrs}}
%The paper’s bibliographic information (its APA citation, typically)
\subsubsection{Identification}
%In around one sentence, how you found it (a Google Scholar search? From a conference’s proceedings? From another paper’s references? %Something else?)

\subsubsection{Summary}
%In around three sentences, a brief, original summary in your own words

\subsubsection{Takeaways}
%In around three sentences, the main takeaways going forward


\subsection{\fullcite{chineseboundary}}
%The paper’s bibliographic information (its APA citation, typically)
\subsubsection{Identification}
%In around one sentence, how you found it (a Google Scholar search? From a conference’s proceedings? From another paper’s references? %Something else?)

\subsubsection{Summary}
%In around three sentences, a brief, original summary in your own words

\subsubsection{Takeaways}
%In around three sentences, the main takeaways going forward

\subsection{\fullcite{ghimire2019comparative}}
%The paper’s bibliographic information (its APA citation, typically)
\subsubsection{Identification}
%In around one sentence, how you found it (a Google Scholar search? From a conference’s proceedings? From another paper’s references? %Something else?)

\subsubsection{Summary}
%In around three sentences, a brief, original summary in your own words

\subsubsection{Takeaways}
%In around three sentences, the main takeaways going forward




\section{Synthesis}
%In about a page, summarize the overall body of work you’ve put together. What are the high-level trends, large takeaways, or open questions you’ve found? If you’ve narrowed in on a particular domain, summarize that domain; if you’re still exploring, discuss the overall direction these efforts are leading you toward. Most importantly, anchor this synthesis in the papers you provided above, citing them where appropriate.



\section{Reflection}
%In about half a page, reflect on the process of finding sources, reading papers, synthesizing their contents, and building your understanding. What was difficult, and what was easy? What are you finding yourself interested in going forward?


\section{Planning}
%In about half a page, provide a plan for what you expect to do next week. What threads or ideas will you pursue? What questions will you seek answers to in the literature?


\printbibliography{}

\end{document}

