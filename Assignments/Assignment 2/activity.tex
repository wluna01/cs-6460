\documentclass[
	%a4paper, % Use A4 paper size
	letterpaper, % Use US letter paper size
]{jdf}

\addbibresource{references.bib}

\author{William Luna}
\email{wluna6@gatech.edu}
\title{Assignment 2: Activity}

\begin{document}
%\lsstyle
% Define custom command for parentheses citation
\newcommand{\pcite}[1]{(\cite{#1})}

\maketitle

\section{Prompt} 

\blockquote{Online education and MOOCs were praised as having the potential to equalize access to education, but critics have suggested that they are having the opposite effect and are disproportionately used by already-affluent audiences. What is the truth about the relationship between online education and equity of access? Is it having an equalizing effect, or is it actually widening the gap in access to education? In answering this, you could choose to consider equity based on gender, race, socioeconomic status, geographic location, or other factors, but you do not need to cover them all.}

\section{Argument}

When William Gibson said "the future is already here, just unevenly distributed," he probably wasn't thinking of a MOOC, considering that Sebastian Thrun's seminal course on AI wouldn't launch for another decade \pcite{gibson, Thrun2012}.

However, Gibson's quote begs an important question for any technology, from fire to fission: if the purpose of invention is to promote human flourishing, how is that flourishing undermined when it also breeds inequity? This essay argues that while MOOCs do disproportionately benefit the already affluent, they are still a net good for society and becoming more equitable over time.

For the first twenty years of the automobile's existence, it was a "hand-crafted, luxury item" that cost around \$30,000 USD\footnote{all financial figures are inflation-adjusted.}, beyond the reach of all but the most affluent Americans. Only with Ford's launch of the Model T in 1908 did the price drop to \$22,000 and by 1925 it cost under \$4,000 \pcite{ford}. The same sharp declines have been seen in solar panels, flat screen TVs, domestic flights, and countless other consumer goods, achieved by economies of scale. For any of these technologies, criticizing them as elitist when they were first invented would have been myopic.\footnote{Not that this has stopped those critics \pcite{StanfordSolarPanels2022}}

Has education–which has been around a lot longer than the flat-screen TV–followed the same trajectory? There are compelling arguments in favor and against. With the cost of undergraduate tuition outpacing inflation since the 1980s, it's easy to argue that college is less affordable than ever \pcite{ForbesTuitionInflation2024}. But with the rise of the MOOC, there's also a compelling case that the cost of education has essentially fallen to zero \pcite{GoodWillHunting1997}.

Unfortunately, cost is far from the only barrier preventing access to MOOCs.

One issue is level of instruction. In most of the world, there is a greater lack of primary school lesson plans than tertiary education, creating a mismatch in the content MOOCs provide and the type of content that would provide the greatest benefit to most communities. Another problem\footnote{Another "problem" contributing to the efficacy of MOOCs, not an actual problem. Low-cost education should always be celebrated.} is that in most countries, higher education has free or low tuition rates, with cost not being the primary barrier to entry \pcite{freeschool}. This means that for most students outside of the United States, access to MOOCs does not impact the affordability of education. Language is another impediment. With most MOOCs' content coming from American universities, the majority of the world lacks the English skills to benefit from their curricula \pcite{Bali2014MOOCPG}. Internet access is yet another obstacle, with  many parts of the world lacking sufficiently cheap and fast bandwidth to engage with online lecture content. In many parts of Africa, downloading a Youtube video costs an average day's wages \pcite{bates2014moocs}.

This has created a status quo where most of the beneficiaries of online learning are those the least in need of it. In the United States, the majority of individuals earning certificates from MOOCs already have a bachelor's degree \pcite{ho2014harvardx}. In Brazil, Russia, India, and China, eighty percent of enrollees were from the wealthiest six percent of households \pcite{adams2013moocs}. Coursera found that completion rates were much higher for the paid versions of its classes, and a Columbia study gathered evidence that minorities from under-represented backgrounds performed worse in online courses than the class as a whole \pcite{solomon2013moocs}. There's also evidence that online courses enforce gender stereotypes around many subjects \pcite{edsurge2014moocs}.\footnote{Even in the case of Georgia Tech's OMSCS, only twenty percent of degree candidates are women \pcite{OMSCSStats2021}.} These observations paint the picture of a paradox, where anyone can access a MOOC, but only those who already come from privileged backgrounds can easily benefit from them.

How did the product and the need become so mismatched? Could the goal of MOOCs have been primarily motivated by financial gain, rather than equity? Probably not. There's substantial evidence that the mission of these initiatives is genuine. Stanford's Honors Cooperative Program was created to accommodate working professionals, OpenCourseWare's goal is to provide global access to world-class education, and EdX has lost money over the majority of its existence \pcite{MITOCWFAQs, Lieberman2018, StanfordRegistrarBulletin}. It would be excessively jaded to presume that these programs were created under anything but altruistic intentions. 

There's an alternative hypothesis to explain the gap between the content MOOCs offer and the needs of their intended audience. Considering that MOOCs originated at elite American universities, is the easiest explanation that the ivory tower is blind to the needs of working class Americans and the developing world? Or was their vision simply never more ambitious than giving the the top ten percent access to the same educational opportunities as the top one percent? All this evidence may make it seem obvious that MOOCs are a force that exacerbates education inequality rather than alleviate it.

However, returning to the opening question, there is ample evidence that the limitations of MOOCs are mostly a limitation of being such a new technology, that there is considerable room for improvement, and that they have already become significantly more equitable over the last decade. Many universities in India have started accepting MOOC certificates of completion as introductory college credit, increasing their value proposition \pcite{hindustan}. Evidence supports that cMOOCs should be prioritized over xMOOCs to increase learner engagement and that synchronous learning provides more emotional support and educational outcomes in primary school \pcite{AdvancedEducation2012, Fabriz2021}. Georgia Tech recently announced a fellowship with Liberia, offering access to the OMSCS program without charging tuition for Liberian citizens \pcite{GeorgiaTech2023}. So not only are the learnings from early MOOCs refining how they are produced and delivered, but policy changes are starting to expand their utility as well.

Criticisms of low completion rates also fail to consider the counterfactual: how many students learned something they otherwise would not have had access to? Dr. Joyner outlines the philosophy of OMSCS on Reddit\footnote{Reading \href{https://www.reddit.com/r/OMSCS/comments/sgg9ne/comment/huwvec7/}{the entire post} is highly encouraged.} by noting that the program "would rather have 10 drop-outs than to have 1 person who would have succeeded if [the program] hadn't rejected them or scared them away" \pcite{JoynerReddit2024}. He also makes the comment that "in a given semester three percent of students are found to have engaged... still, at our scale, that's a lot of actual people" \pcite{Joyner2024}.\footnote{This comment was originally in the context of academic misconduct, but the observation relevant.} Both these observations highlight that key indicator of a MOOC's impact should be the absolute number of participants that complete it, not the percentage.

Furthermore, whatever costs are associated with MOOCs, however low, may be necessary to increase motivation and completion rates. Coursera found that students who paid at least \$30 to enroll in a course had a significantly higher pass rate than students who paid nothing \pcite{solomon2013moocs}.\footnote{Although \href{https://xkcd.com/552/}{correlation is not causation}.} Microsoft employees in India attribute their professional success to their access to MOOCs \pcite{agarwal2016moocs}. There is also a misconception that MOOCs seek to disrupt or replace traditional higher education, when they mostly seek to complement it \pcite{kim2014moocs}.\footnote{The institutions that antagonize higher education the most, like the Thiel Fellowship, usually do not produce their own educational content.}

In conclusion, MOOCs do indeed have a counter-intuitive effect on the educational system, delivering the most benefit to the already wealthy, privileged, and well-educated. However, there's equal evidence that like all new technology, it is rapidly disseminating its benefits to the rest of the world. Criticisms of its current state are valid, but not grounds to kill the idea in the cradle. In the words of Sebastian Thrun, "to all those who declared our experiment a failure, you have to understand how innovation works. Few ideas work on the first try. Iteration is key to innovation" \pcite{adams2013moocs}.

\printbibliography{}

\end{document}

