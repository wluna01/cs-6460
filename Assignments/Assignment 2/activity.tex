\documentclass[
	%a4paper, % Use A4 paper size
	letterpaper, % Use US letter paper size
]{jdf}

\addbibresource{references.bib}

\author{William Luna}
\email{wluna6@gatech.edu}
\title{Assignment 2: Activity}

\begin{document}
%\lsstyle
% Define custom command for parentheses citation
\newcommand{\pcite}[1]{(\cite{#1})}

\maketitle

\section{Prompt} 

\blockquote{Online education and MOOCs were praised as having the potential to equalize access to education, but critics have suggested that they are having the opposite effect and are disproportionately used by already-affluent audiences. What is the truth about the relationship between online education and equity of access? Is it having an equalizing effect, or is it actually widening the gap in access to education? In answering this, you could choose to consider equity based on gender, race, socioeconomic status, geographic location, or other factors, but you do not need to cover them all.}

\section{Argument}

When William Gibson said "the future is already here, just unevenly distributed," he probably wasn't thinking of a MOOC, considering that Sebastian Thrun's seminal course on AI wouldn't launch until a decade after he said it \pcite{gibson, Thrun2012}.

However, Gibson's quote begs an important question for any technology, from fire to fission: given that the assumed purpose of any new invention is to promote human flourishing, how is that flourishing undermined when it also breeds inequity?

For any technology, it \textit{is} a matter of when, not if, it will cause that inequity. And massively online open courses are no different. This essay argues that while MOOCs do disproportionately benefit the already affluent, they are more equitable than most technologies, and have become more equitable over time.

For the first twenty years of the automobile's existence, it was a "hand-crafted, luxury item" that cost around \$30,000 USD\footnote{all financial figures are inflation-adjusted.}, beyond the reach of all but the most affluent Americans. Only with Ford's launch of the Model T in 1908 did the price drop to \$22,000 and by 1925 it cost under \$4,000 \pcite{ford}. The same sharp declines have been seen in solar panels, flat screen TVs, domestic flights, and countless other consumer goods, achieved by economies of scale. For any of these technologies, criticizing them as elitist when they were first invented would have been myopic.\footnote{Not that this has stopped those critics \pcite{StanfordSolarPanels2022}}

Has education–which has been around a lot longer than the flat-screen TV–followed the same trajectory? There are compelling arguments in favor and against. With the cost of undergraduate tuition outpacing inflation since the 1980's, it's easy to argue that college is less affordable than ever \pcite{ForbesTuitionInflation2024}. But with the rise of the MOOC, there's an equally compelling case that the cost of education has essentially fallen to zero \pcite{GoodWillHunting1997}.




The thoughtful reader will observe that education is not a flat screen TV. Education has been around much longer than all of the aforementioned technologies, but despite the head start, the cost of higher education in the United States has outpaced inflation since the 1980s \pcite{collegeboard}. While an examination into the cause of this increase is beyond the scope of this essay, a reductive explanation is that \textbf{education has primarily been conceived of as a service, not a good, making it more difficult to take advantage of the same economic levers that drop the prices of most technologies over time.}

With this framing in mind, it becomes possible to frame a MOOC as a means of converting education primarily being a service to a good, that can take advantage of economies of scale, such as spending thousands producing lecture videos but showing those videos to hundreds of thousands of students, lowering costs.

However, few claim that cost is the barrier preventing access to MOOCs. One issue is level of instruction. In most of the world, there is a greater lack of primary school content than tertiary education \pcite{bates2014moocs}. Another is policy differences. In most countries, higher education has free or low tuition rates, with cost not being the primary barrier to entry \pcite{freeschool}. With most MOOCs' content in English, the majority of the world lacks the language skills to benefit from their curricula \pcite{Bali2014MOOCPG}. And in many cases, access to MOOCs are limited not to mention their lack of infrastructure to take advantage of bandwidth consuming content.

Given the evidence above, it seems less that MOOCs are failing to achieve their aims at promoting educational equity, and more that their primary goal is simply not promoting equity. 

Considering that MOOCs originated in the ivory tower of elite American universities, particularly Stanford's Honors Cooperative Programe, MIT's OpenCourseWare, and Harvard's EdX, this is hardly surprising. There's substantial evidence that the stated mission of these initiatives is genuine. HCP was created to accommodate working professionals, OpenCourseWare's stated mission is to provide global access to world-class education, and EdX has lost money over the majority of its existence \pcite{MITOCWFAQs, Lieberman2018, StanfordRegistrarBulletin}. It would be excessively jaded to presume that these programs were created under anything but altruistic intentions.

However, a more reasonable hypothesis is that these organizations, despite those intentions, were divorced from the needs of the learners who stood to benefit from their efforts.

\begin{enumerate}
    \item Provide famous historical examples of slow to diffuse technologies that we use every day. Flight. Cars. 
    \item Provide examples of MOOCs also being unevenly distributed
    \item Provide examples examples of other educational technologies that are also diffusing.
    \item Provide examples that they are on a trajectory of diffusing
    \item Anecdote of ambiguity in the San Francisco algebra middle school debate
    \item Return to Gibson quote, point out that we might expand it to "the future is already somewhere, but it will make it here eventually." Bring up the arc of history bends toward justice. 
\end{enumerate}


\section{Research Notes}

%Also the priority in most Third World countries is not for courses from high-level Stanford University professors, but for programs for elementary and high schools. Finally, while mobile phones are widespread in Africa, they operate on very narrow bandwidths. For instance, it costs US$2 to download a typical YouTube video – equivalent to a day’s salary for many Africans. Streamed 50 minute video lectures then have limited applicability.

Others make argument that MooCs don't address the more pressing need in the developing world for primary education materials, not to mention their lack of infrastructure to take advantage of bandwidth consuming content. And in many countries public higher education is free, meaning MooCs are only beneficial if they can surpass that already high bar.
(\cite{bates2014moocs}).

%In a research report from Ho et al. (2014), researchers at Harvard University and MIT found that on the first 17 MOOCs offered through edX, 66 per cent of all participants, and 74 per cent of all who obtained a certificate, had a bachelor’s degree or above, 71 per cent were male, and the average age was 26. This and other studies also found that a high proportion of participants came from outside the USA, ranging from 40-60 per cent of all participants, indicating strong interest internationally in open access to high quality university teaching.

%file:///Users/wluna/Downloads/SSRN-id2381263.pdf

Most already have a bachelor's degree. But fails to consider the counterfactual. How many people in absolute numbers learned something they otherwise would not have had access to? \cite{ho2014harvardx}

%Even something as simple as having skin in the game can make students feel more engaged. Most MOOCs are free, so students don't feel a financial bite if they drop a course or perform poorly. Coursera found that students who pay $30 to $90 for the company's Signature Track identity-verification program, which confirms that they took a course and passed, are substantially more likely to finish the course.
Paying just something to have skin in the game increases completion rates. 

%It runs counter to the access-for-all mission of MOOCs, but evidence is mounting that online learning doesn't work for all students. The Columbia study of Washington community-college students found that all students performed less well in online courses than in face-to-face ones, but the gap was even wider among those with lower GPAs, men and African-Americans.

Potential to exacerbate inequalities since populations who struggle continue to do so.\cite{solomon2013moocs}


%Gender bias in different subjects: The percentage of females taking classes on different subjects “seems to reinforce stereotypes with regard to which classes are ‘more male’ or ‘more female.’” For instance, females make up the majority of students in classes on Food and Nutrition, Teacher Professional Development and Medicine. But they make up less than 20% of students in computer science and engineering.
Gender bias \cite{edsurge2014moocs}

%some 80 percent of MOOC users around the world already have an advanced degree, casting some doubt on the democratizing notion. For that study, Penn surveyed 35,000 students in more than 200 countries who took 32 different Penn MOOCs.  Around the world, the study found, it was the economic elite who were taking MOOCs. In Brazil, Russia, India, China and South Africa, 80percent of MOOC students came from the richest 6percent of the population. Of course MOOC students need to have access to computers with fast Internet connections, which automatically throws up a bar to the poor.
Mostly for the elite at first blush, but Sebastian Thrun quote that “To all those who declared our experiment a failure, you have to understand how innovation works,” the Times quotes from Thrun’s blog. “Few ideas work on the first try. Iteration is key to innovation.” \cite{adams2013moocs}.

%Myth #2 - The people involved in MOOCs think that open online education will replace traditional higher education:
No one is trying to be disruptive even if outsiders perceive it that way \cite{kim2014moocs}.

%Let me share a few examples of students across the globe who’ve experienced exactly that. Akshay, a learner from India, credits his job at Microsoft to the experience he gained while taking edX courses. Akshay turned to edX after struggling to stay interested and engaged at his local college. After taking numerous MIT courses on edX, even participating as a community TA, Akshay’s level of experience and knowledge was impressive enough to secure him a job at Microsoft in India.
Anecdotal evidence that students in foreign countries benefit. But feels like PR move
\cite{agarwal2016moocs}

%moocs not doing well abroad https://docs.edtechhub.org/lib/WBWECXQ5


\printbibliography{}

\end{document}

