\documentclass[
	%a4paper, % Use A4 paper size
	letterpaper, % Use US letter paper size
]{jdf}

\addbibresource{references.bib}

\author{William Luna}
\email{wluna6@gatech.edu}
\title{Assignment 2: Activity}

\begin{document}
%\lsstyle

\maketitle

\section{Prompt} 

\blockquote{Online education and MOOCs were praised as having the potential to equalize access to education, but critics have suggested that they are having the opposite effect and are disproportionately used by already-affluent audiences. What is the truth about the relationship between online education and equity of access? Is it having an equalizing effect, or is it actually widening the gap in access to education? In answering this, you could choose to consider equity based on gender, race, socioeconomic status, geographic location, or other factors, but you do not need to cover them all.}

\section{Argument}

Research has drawn a distinction between two types of MooCs, xMoocs and cMoocs and bring up problems in accessbility to audiences who don't speak English (\cite{Bali2014MOOCPG}).

%Also the priority in most Third World countries is not for courses from high-level Stanford University professors, but for programs for elementary and high schools. Finally, while mobile phones are widespread in Africa, they operate on very narrow bandwidths. For instance, it costs US$2 to download a typical YouTube video – equivalent to a day’s salary for many Africans. Streamed 50 minute video lectures then have limited applicability.

Others make argument that MooCs don't address the more pressing need in the developing world for primary education materials, not to mention their lack of infrastructure to take advantage of bandwidth consuming content. And in many countries public higher education is free, meaning MooCs are only beneficial if they can surpass tht already high bar.
(\cite{bates2014moocs}).

%In a research report from Ho et al. (2014), researchers at Harvard University and MIT found that on the first 17 MOOCs offered through edX, 66 per cent of all participants, and 74 per cent of all who obtained a certificate, had a bachelor’s degree or above, 71 per cent were male, and the average age was 26. This and other studies also found that a high proportion of participants came from outside the USA, ranging from 40-60 per cent of all participants, indicating strong interest internationally in open access to high quality university teaching.

%file:///Users/wluna/Downloads/SSRN-id2381263.pdf

Most already have a bachelor's degree. But fails to consider the counterfactual. How many people in absolute numbers learned something they otherwise would not have had access to? \cite{ho2014harvardx}

%Even something as simple as having skin in the game can make students feel more engaged. Most MOOCs are free, so students don't feel a financial bite if they drop a course or perform poorly. Coursera found that students who pay $30 to $90 for the company's Signature Track identity-verification program, which confirms that they took a course and passed, are substantially more likely to finish the course.
Paying just something to have skin in the game increases completion rates. 

%It runs counter to the access-for-all mission of MOOCs, but evidence is mounting that online learning doesn't work for all students. The Columbia study of Washington community-college students found that all students performed less well in online courses than in face-to-face ones, but the gap was even wider among those with lower GPAs, men and African-Americans.

Potential to exacerbate inequalities since populations who struggle continue to do so.\cite{solomon2013moocs}


%Gender bias in different subjects: The percentage of females taking classes on different subjects “seems to reinforce stereotypes with regard to which classes are ‘more male’ or ‘more female.’” For instance, females make up the majority of students in classes on Food and Nutrition, Teacher Professional Development and Medicine. But they make up less than 20% of students in computer science and engineering.
Gender bias \cite{edsurge2014moocs}

%some 80 percent of MOOC users around the world already have an advanced degree, casting some doubt on the democratizing notion. For that study, Penn surveyed 35,000 students in more than 200 countries who took 32 different Penn MOOCs.  Around the world, the study found, it was the economic elite who were taking MOOCs. In Brazil, Russia, India, China and South Africa, 80percent of MOOC students came from the richest 6percent of the population. Of course MOOC students need to have access to computers with fast Internet connections, which automatically throws up a bar to the poor.
Mostly for the elite at first blush, but Sebastian Thrun quote that “To all those who declared our experiment a failure, you have to understand how innovation works,” the Times quotes from Thrun’s blog. “Few ideas work on the first try. Iteration is key to innovation.” \cite{adams2013moocs}.

%Myth #2 - The people involved in MOOCs think that open online education will replace traditional higher education:
No one is trying to be disruptive even if outsiders perceive it that way \cite{kim2014moocs}.

%Let me share a few examples of students across the globe who’ve experienced exactly that. Akshay, a learner from India, credits his job at Microsoft to the experience he gained while taking edX courses. Akshay turned to edX after struggling to stay interested and engaged at his local college. After taking numerous MIT courses on edX, even participating as a community TA, Akshay’s level of experience and knowledge was impressive enough to secure him a job at Microsoft in India.
Anecdotal evidence that students in foreign countries benefit. But feels like PR move
\cite{agarwal2016moocs}

%moocs not doing well abroad https://docs.edtechhub.org/lib/WBWECXQ5


\printbibliography{}

\end{document}

