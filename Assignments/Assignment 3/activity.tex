\documentclass[
	%a4paper, % Use A4 paper size
	letterpaper, % Use US letter paper size
]{jdf}

\addbibresource{references.bib}
\newcommand{\pcite}[1]{(\cite{#1})}

\author{William Luna}
\email{wluna6@gatech.edu}
\title{Assignment 3: Activity}

\begin{document}
%\lsstyle
% Define custom command for parentheses citation
\newcommand{\pcite}[1]{(\cite{#1})}

\maketitle
%INSTRUCTIONS
%In the Project Proposal, you will state a problem or question, and then give a plan for how you will solve or answer it. The mistake many people make at this stage is to rush to what they plan to do without pausing to thoroughly define the problem or question itself. In the absence of a good problem statement or research question, it is difficult or impossible to judge whether the solution or answer adequately addressed the problem.

%So, for this activity, you are going to practice writing both a problem statement and a set of research questions. A problem statement addresses Development and Content track projects: there exists some problem that needs to be solved (where that problem could be, “people need to learn X, but at present they cannot”). Research questions address Research track projects: there is a phenomenon that needs to be explained, or relatedly, there is an area in which phenomena may exist that have not yet been identified.

%Although you’ll generally choose one track for your project (although there may be overlaps), in this activity you’ll practice writing both. This will equip you to give better feedback to your classmates. For Activity 3, you’ll thus turn in both a problem statement and a set of research questions.

%Ideally, you are far along enough in your research that you can write problem statements and research questions that may be rough drafts for what you eventually use in your proposal. If not, you’re welcome to choose any problem with which you’re familiar. It does not even have to be in education for this activity (although we would recommend staying topical) but it may be easier to write in a domain with which you’re more familiar. Don’t overcomplicate what counts as a ‘problem’ or ‘phenomenon’: a problem is anything that isn’t working as well as it could, and a phenomenon is anything we can observe and may want to explain or explore. “Students need grades and feedback faster” or “Professional certifications are prohibitively expensive” would both be problems. “Retention rates in online courses are low” or “We do not know about the structures of online courses” would be phenomena to explore.


\section{Problem Statement}

%There are many ways to write a problem statement, but in order to give you a starting point, we follow the structure advocated by Ashford University

%Links to an external site. among others. You might not follow this structure exactly in your proposal, but following it now should give you good practice on the value in defining these details piece-by-piece. Your problem statement, which defines a problem to be solved, should include the following parts:

\subsection{Background Information}
    %Background Information. First, provide some background information. Depending on your problem area, the reader may not be familiar with its basic vocabulary and existing structures. Provide enough background that someone with limited familiarity with the area will be able to understand the general problem.
Learning vocabulary is a critical part of developing fluency in a foreign language. This is likely a self-evident statement to both lay-people and experts, since learning a foreign language is a near-universal requirement in global education. However,

\subsection{General Problem Statement}
    %General Problem Statement. The general problem statement describes a broad problem within the domain you described above. The problem here is likely so general as to be unsolvable without further specification. For example, “global temperatures are rising” is a general problem statement. It is a stated problem, but without knowing more about why the problem exists, it is not solvable.
Learning vocabulary of a foreign language is a difficult process.

\subsection{Scholarly Support}
    %Scholarly Support. Here, you would provide evidence that the problem or phenomenon actually exists. Note that if scholarly support is absent, you may supply other forms of support, although a complete lack of scholarly support means you would likely first approach this as a research question to establish if the problem exists in the first place.
An analysis suggests that a reader needs to know at least the five thousand most common English words in order to be able to read young adult fiction \pcite{nation1992vocabulary}. It takes a median of several months for a student to commit a word to long-term memory with spaced repetition flashcards \pcite{Murre2015ReplicationAA}. There are other more engaging means of acquiring vocabulary, but these take even longer without additional study materials \pcite{free_voluntary_reading}. These observations contribute to the fact that the average student of Mandarin Chinese needs 2200-3000 hours of study to acquire enough vocabulary to read authentic texts unassisted, meaning that most university majors in the language will not be able to reach this milestone in four years \pcite{Liu2015AnAO}.

Therefore,
\subsection{Specific Problem Statement}
    %Specific Problem Statement. Here, based on that scholarly support, you drill the problem down into details that can actually be solvable. For example, “Industry is outputting carbon emissions at a greater rate than can be absorbed by the earth”, “The earth is retaining greenhouse gases causing an increasing concentration over time”, or “Environmentalism tends to be prioritized only by affluent nations” would all be more specific ways to state the problem: these are more solvable. You may find you define your problem statement specifically in a way that connects to the solution that you have in mind; that’s okay.
Current methods of acquiring vocabulary in a foreign language require a trade-off between interesting but inefficient methods, such as extensive reading, versus highly-effective but boring methods such as drilling flash cards using spaced repetition software.

\subsection{Closing Commentary}
    %Closing Commentary. Finally, you would briefly discuss the overall impact of the problem you have described. How will society be affected if it remains unsolved? How will it be affected if it’s solved?
Any barrier to learning a foreign language means that fewer individuals will achieve fluency in the language they are studying, or might have achieved a higher level of fluency, or might take longer to achieve that same level of fluency. 

Is a world where fewer people can speak more than one language any worse off than a more multilingual one? That is a subjective question. With the advent of machine translation and large language models, one could argue that the opportunity cost of studyng a foreign language has plummeted in the last decade. Many universities in the United States have cut funding for their language programs, with one going so far as to do away with the foreign languages department entirely \pcite{wvmetronews2023}.

However, this problem statement rests on a firm disagreement with that perspective. It rests on the argument that human connection can only flourish when we are able to speak across cultures and continents directly, without some digital intermediary.
%We expect a good problem statement to be around 2 pages in JDF. This is neither a minimum nor a maximum, but rather is just a heuristic to understand the level of depth we would expect. Ignore the length heuristics from Ashford University itself; we expect more depth.

\section{Research Questions}

\subsection{Question}

\subsection{Sub-Question 1:}

\subsection{Sub-Question 2:}

\subsection{Sub-Question 3:}

%While problem statements focus on problems to be solved, research statements focus on phenomena to be observed or explained. Research questions are generally expected

%Links to an external site. to have certain characteristics:

    %Clarity: Research questions should be clear and specific enough that the audience can understand the purpose.
    %Focused: Research questions should have a sufficiently narrow focus as to be addressable and answerable.
    %Concise: To be clear and focused, research questions are also expected to use as few words as possible.
    %Complex: Research questions generally cannot be answered by simple numbers or yes/nos; questions like ‘how’ and ‘why’ lead to more complex answers.
    %Arguable: Research questions should be addressable by facts rather than opinions.
    %Hierarchical: Research questions can generally be decomposed into sub-questions which, if answered, will supply an answer to the overall question.

%Write a research question that can be decomposed into at least three smaller questions. For example, the question, “How can AI be used to improve performance on algebra homework?” could be decomposed into, “To what extent can AI make sense of students’ intermediate problem-solving steps?”, “To what extent can AI use that understanding to generate hints?”, and “To what extent do such hints improve students’ performance?”

%Then, justify that all three sub-questions are meet the criteria above for complexity and arguability (clarity, focus, and conciseness will be relatively self-evident). What kinds of answers can you expect to receive to these questions, and what kinds of facts or data will support those answers?

%Note that this is expected to be a difficult exercise; do not expect it to come naturally. Writing good research questions is difficult to do, but it is a very important skill to learn. A quick Google search for “how to write good research questions” will bring up some additional valuable material.

%We expect a good set of research questions with accompanying justification to be around 2 pages in JDF. This is neither a minimum nor a maximum, but rather is just a heuristic to understand the level of depth we would expect.

\printbibliography{}

\end{document}

