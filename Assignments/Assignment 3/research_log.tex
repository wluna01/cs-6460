\documentclass[
	%a4paper, % Use A4 paper size
	letterpaper, % Use US letter paper size
]{jdf}
\newcommand{\pcite}[1]{(\cite{#1})}
\addbibresource{../../references.bib}
\author{William Luna}
\email{wluna6@gatech.edu}
\title{Assignment 3: Research Log}

\begin{document}
%\lsstyle

\maketitle

\section{Background}
% In about half a page, summarize your current state. This would largely cover where you left off last week.
The research from last week brought a great deal of clarity to what I want to focus on. To summarize the core learnings of the last two weeks:
\begin{enumerate}
    \item Vocabulary acquisition is an important part of language learning,
    \item Extensive Reading is a helpful exercise in acquiring new vocabulary,
    \item Using a dictionary during extensive reading has a neutral to positive impact on acquiring new vocabulary,
    \item New vocabulary is most efficiently transferred into long-term memory when prompted for recall at increasing intervals, and
    \item Extensive reading is a potential vehicle for prompting recall of newly acquired vocabulary.
\end{enumerate}

This is a very formal way of pointing out what is perhaps quite obvious: we read books, look up new words when we don't recognize them, and can bet that the same book will use those new words enough to cement them into our long-term memory.    

Almost all of the papers reviewed in the last two weeks either provide evidence for one of the numbered statements above, or address a challenge in implementing them, such as "how to best provide extensive reading material to beginners?", "are monolingual or bilingual dictionaries more effective?", or "what is the most efficient algorithm for spaced repetition?" 

NEEDS SECTION ON WHAT I HOPE TO GET OUT OF PAPERS BELOW.

\section{Papers and Other Reference Material}
%As you walk through the Research Guide, you’ll be finding lots of papers to read. Here, you’ll make a list of the papers you come across and give considerable attention to. We would expect the Research Log to include at least 15-20 sources (though more is fine as well), and at least 12 (preferably more) should be academic and peer-reviewed. You may include blog posts, newspaper articles, etc. as well, but you should have at least 12 academic sources, too.

\subsection{\fullcite{ye2023storypark}}
%The paper’s bibliographic information (its APA citation, typically)
\subsubsection{Identification}
%In around one sentence, how you found it (a Google Scholar search? From a conference’s proceedings? From another paper’s references? %Something else?)
Asked Chat GPT to provide papers that leverage LLMs for language learning applications. Interestingly, it hallucinated, providing the name of a paper that doesn't exist, but this was the top result returned by Google when typing in the fake paper's name.

\subsubsection{Summary}
%In around three sentences, a brief, original summary in your own words
This paper evaluates the capacity of GPT-3 to enable interactive storytelling with children. A simple storytelling AI is constructed with prompt engineering wrapped around the GPT, and it is embedded into an iPad application where a child is asked to provide details about a story that inform the future narrative. An experiment is administered where children interact with two versions of the app, one where story continuity is generated by the GPT and another where it is generated by a human. Child develpment experts rate the resulting stories on indices of creativity, interaction, comprehensibility, and other factors. The GPT is found to perform only slightly worse than the human storytellers.

\subsubsection{Takeaways}
%In around three sentences, the main takeaways going forward
This is one of the first papers I've read that leverage a GPT to incorporate to a reader/listener's feedback into the subsequent parts of the story it is telling. However, it does not consider language learning a core component of the research. My main takeaway is actually the use of solely qualitative feedback in the papers analysis, results, and conclusions section. While I may agree that the qualitative feedback of human experts is superior to an assessment generated by a machine learning model, evaluation metrics for LLM output were well-defined when this paper was published, making the reason for their absence unclear.

\subsection{\fullcite{llmtutor}}
%The paper’s bibliographic information (its APA citation, typically)
\subsubsection{Identification}
%In around one sentence, how you found it (a Google Scholar search? From a conference’s proceedings? From another paper’s references? %Something else?)
I made a generic Google Scholar search exploring how LLMs have most recently been leveraged in the context of language learning.

\subsubsection{Summary}
%In around three sentences, a brief, original summary in your own words
This paper includes a literature review, a refresher on multiple theories of learning, introduces different prompts to train an LLM at several tasks, uses an evaluation framework to grade the LLMs' performance, runs an experiment, and evaluates the results along quantitative and qualitative benchmarks. All in six pages. 

\subsubsection{Takeaways}
%In around three sentences, the main takeaways going forward
I felt this paper was rather impenetrable, my biggest takeaway being that the importance of a paper's length matching the ambition of its scope. The consequence was a paper with no obvious conclusions. This paper being on arxiv, it's possible that it's still in development. If so, this is a cautionary tale about sacrificing peer review in order to read the most recent papers that discuss bleeding-edge LLMs.

\subsection{\fullcite{vygotsky}}
%The paper’s bibliographic information (its APA citation, typically)
\subsubsection{Identification}
%In around one sentence, how you found it (a Google Scholar search? From a conference’s proceedings? From another paper’s references? %Something else?)
Vygotsky's theory of the Zone of Proximal Development came up in the literature overview of the paper above and Chapter Six of this book is where the term was coined.

\subsubsection{Summary}
%In around three sentences, a brief, original summary in your own words
The chapter begin with an overview of competing definitions of learning, development, and how the two concepts interact, especially in regards to early childhood education. Vygotsky proceeds to reject all previously discussed theories in favor of the Zone of Proximal Development, which he defines as "\textit{the distance between the actual developmental level as determined by independent problem solving and the level of potential development as determined through problem solving under adult guidance or in collaboration with more capable peers}" (emphasis his). There's additional discussion of how development of children is radically different before and after they are of schooling age.

\subsubsection{Takeaways}
%In around three sentences, the main takeaways going forward
Vygotsky's theory interests me from the perspective that the Zone of Proximal Development appears to be analogous to, or at least complementary to, Krashen's theory of Massive Comprehensible input (i+1). Both rest make the claim that learning is best performed when students are asked to reach \textit{just} beyond their cognitive grasp. However, it's unclear if this conclusion has consensus in the scientific community or is my own unique conclusion biased from limited exposure to the field.

\subsection{\fullcite{vygostky_krashen_not_same}}
%The paper’s bibliographic information (its APA citation, typically)
\subsubsection{Identification}
%In around one sentence, how you found it (a Google Scholar search? From a conference’s proceedings? From another paper’s references? %Something else?)
I wanted to pull on the thread of the above takeaway and asked Chat GPT to provide papers that explore the similarities and differences between Krashen's and Vygotsky's theories.

\subsubsection{Summary}
%In around three sentences, a brief, original summary in your own words
The paper covers the history of comparisons between Krashen's i+1 and Vygotsk''s ZPD, from a 1983 paper that directly conflates the two ("'near the student's "zone of proximal development" or "i+1"'") up through a (then contemporary) 1996 paper that also equates them ("'the i+1 stage is equivalent to Vygotsky's zone of proximal development'"). The author then deems the two theories "superficially similar" but "profoundly different" mostly on the grounds that Krashen and Vygotsky had profoundly different views on how learning intersects with development, and that Krashen's theories don't transfer well to domains outside of language acquisition.

\subsubsection{Takeaways}
%In around three sentences, the main takeaways going forward
This is a fascinating thread of academic inquiry, but I'm going to stop here since further understanding of the nuance won't do much to inform the direction of my project. The major learning of this argument is that Vygotsky's frameworks shouldn't be assumed to relate to Krashen's by default, meaning that I should not dig into current literature on his work as a new font of complementary ideas.

\subsection{\fullcite{controllable_story_generation}}
%The paper’s bibliographic information (its APA citation, typically)
\subsubsection{Identification}
%In around one sentence, how you found it (a Google Scholar search? From a conference’s proceedings? From another paper’s references? %Something else?)
I asked Chat GPT the following:

\blockquote{I have an idea to leverage an LLM to continuously generate new parts of a story, where a key goal of the LLM is to 1) produce text that is at the reader's level and 2) uses new vocabulary in the story according to a spaced repetition program. To elaborate, the idea is that the reader clicks on which words they do not understand, which are then added to word bank, where each word in this word bank should be presented at the ideal interval in the text generated by the LLM.

I know LingQ as a product provides well graded reading content and can add unknown words to a flashcard deck. But to my knowledge, it and all other products and research are missing the component of dynamically incorporating those unknown words into an evolving story.}

It provided this research paper as a relevant resource while admitting that it does not have an identical goal.

\subsubsection{Summary}
%In around three sentences, a brief, original summary in your own words
This paper proposes the idea of using an LLM to dynamically generate a story based on user input. The introduction walks through the history of the field from Meehan's TALE-SPIN through LSTM-based that even take advantage of Ilya Sutskever's work on encoder-decoder neural networks, the groundwork for Chat GPT. \textit{Controllability} is discussed as an under-examined aspect of AI-based story generation. A model is trained to capture how a story ends-happy, sad, or ambiguous–and is deployed to provide different endings to several unfinished stories.

\subsubsection{Takeaways}
%In around three sentences, the main takeaways going forward
This is a critically helpful paper in my research. It points out that I have under-explored story generation as a body of research. It will be important to see if more recent papers that cite this one more heavily leverage LLMs and/or successfully provide more granual story controls than the emotion of the ending.

\subsection{\fullcite{recent_story_generation_review}}
%The paper’s bibliographic information (its APA citation, typically)
\subsubsection{Identification}
%In around one sentence, how you found it (a Google Scholar search? From a conference’s proceedings? From another paper’s references? %Something else?)
I pressed Chat GPT on a paper it had probably hallucinated. It offered this as an alternative.

\subsubsection{Summary}
%In around three sentences, a brief, original summary in your own words
This paper provides strong evidence that the latest language models are not only effective at concluding and summarizing stories, but that humans prefer their output over smaller models purpose-built for the task. The paper highlights some challenging edge cases, such as the model occasionally using expletives, responding in a foreign language, or soft plagiarizing existing literary works.

\subsubsection{Takeaways}
%In around three sentences, the main takeaways going forward
This paper surfaces the pitfalls of building any system that generates stochastic output, in that no matter how well it performs in general, there's no guarantee that it won't generate problematic output in a specific instance.

\subsection{\fullcite{ai_human_taking_turns_creating_story}}
%The paper’s bibliographic information (its APA citation, typically)
\subsubsection{Identification}
%In around one sentence, how you found it (a Google Scholar search? From a conference’s proceedings? From another paper’s references? %Something else?)
Chat GPT offered this paper as an additional alternative to the one it hallucinated.

\subsubsection{Summary}
%In around three sentences, a brief, original summary in your own words
Researchers fine-tune a publicly-available LLM on top of writing prompts and corresponding fictional works generating from those prompts. This LLM is then deployed as a game, where it starts the story, waits for a human to continue, the story, adds to the story itself, and repeats this process until the human concludes the story. Mechanical Turk workers were asked to rate the continuity of stories generated by both the fine-tuned version of the model and the default LLM, soliciting pairwise user preferences. There is strong evidence that the fine-tuned model generated output most preferred by humans.

\subsubsection{Takeaways}
%In around three sentences, the main takeaways going forward
This paper's use of story starters echoes my similar idea to start all stories with one of a handful of introductions. It also highlights the challenge in establishing useful evaluation frameworks. In my opinion, the evaluation framework mismatches the goal of the study, since proving that the fine-tuned model outperforms the others does not explicitly validate a user's apetite or overall enthusiasm for the exercise of creating a story together with an AI.

 \subsection{\fullcite{}}
%The paper’s bibliographic information (its APA citation, typically)
\subsubsection{Identification}
%In around one sentence, how you found it (a Google Scholar search? From a conference’s proceedings? From another paper’s references? %Something else?)


\subsubsection{Summary}
%In around three sentences, a brief, original summary in your own words


\subsubsection{Takeaways}
%In around three sentences, the main takeaways going forward

 \subsection{\fullcite{}}
%The paper’s bibliographic information (its APA citation, typically)
\subsubsection{Identification}
%In around one sentence, how you found it (a Google Scholar search? From a conference’s proceedings? From another paper’s references? %Something else?)


\subsubsection{Summary}
%In around three sentences, a brief, original summary in your own words


\subsubsection{Takeaways}
%In around three sentences, the main takeaways going forward

 \subsection{\fullcite{}}
%The paper’s bibliographic information (its APA citation, typically)
\subsubsection{Identification}
%In around one sentence, how you found it (a Google Scholar search? From a conference’s proceedings? From another paper’s references? %Something else?)


\subsubsection{Summary}
%In around three sentences, a brief, original summary in your own words


\subsubsection{Takeaways}
%In around three sentences, the main takeaways going forward

 \subsection{\fullcite{}}
%The paper’s bibliographic information (its APA citation, typically)
\subsubsection{Identification}
%In around one sentence, how you found it (a Google Scholar search? From a conference’s proceedings? From another paper’s references? %Something else?)


\subsubsection{Summary}
%In around three sentences, a brief, original summary in your own words


\subsubsection{Takeaways}
%In around three sentences, the main takeaways going forward

 \subsection{\fullcite{}}
%The paper’s bibliographic information (its APA citation, typically)
\subsubsection{Identification}
%In around one sentence, how you found it (a Google Scholar search? From a conference’s proceedings? From another paper’s references? %Something else?)


\subsubsection{Summary}
%In around three sentences, a brief, original summary in your own words


\subsubsection{Takeaways}
%In around three sentences, the main takeaways going forward

 \subsection{\fullcite{}}
%The paper’s bibliographic information (its APA citation, typically)
\subsubsection{Identification}
%In around one sentence, how you found it (a Google Scholar search? From a conference’s proceedings? From another paper’s references? %Something else?)


\subsubsection{Summary}
%In around three sentences, a brief, original summary in your own words


\subsubsection{Takeaways}
%In around three sentences, the main takeaways going forward

 \subsection{\fullcite{}}
%The paper’s bibliographic information (its APA citation, typically)
\subsubsection{Identification}
%In around one sentence, how you found it (a Google Scholar search? From a conference’s proceedings? From another paper’s references? %Something else?)
First result when searching "combining flashcards with extensive reading" on Google.

\subsubsection{Summary}
%In around three sentences, a brief, original summary in your own words
This article does a wonderful job laying out the trade-offs of acquiring vocabulary through extensive reading (in context, slow) vs studying flashcards (out of context, fast). It poses interlinear reading as a best of both worlds solution, where it's possible to read in the target language but with a translation in one's native language in smaller font below each word.

\subsubsection{Takeaways}
%In around three sentences, the main takeaways going forward
I take this blog post as further validation that combining spaced repetition with extensive reading is an acknowledged problem in the language learning community. It's also reassuring to see that there is no elegant proposal to blend both techniques together that approximates my project proposal.

\subsubsection{Identification}
%In around one sentence, how you found it (a Google Scholar search? From a conference’s proceedings? From another paper’s references? %Something else?)


\subsubsection{Summary}
%In around three sentences, a brief, original summary in your own words


\subsubsection{Takeaways}
%In around three sentences, the main takeaways going forward

\section{Synthesis}
%In about a page, summarize the overall body of work you’ve put together. What are the high-level trends, large takeaways, or open questions you’ve found? If you’ve narrowed in on a particular domain, summarize that domain; if you’re still exploring, discuss the overall direction these efforts are leading you toward. Most importantly, anchor this synthesis in the papers you provided above, citing them where appropriate.


\section{Reflection}
%In about half a page, reflect on the process of finding sources, reading papers, synthesizing their contents, and building your understanding. What was difficult, and what was easy? What are you finding yourself interested in going forward?


\section{Planning}
%In about half a page, provide a plan for what you expect to do next week. What threads or ideas will you pursue? What questions will you seek answers to in the literature?

%https://www.microsoft.com/en-us/research/wp-content/uploads/2004/06/DynamicLLearning.pdf
%https://www.sciencedirect.com/science/article/pii/S0167639309000430

\printbibliography{}

\end{document}

