\documentclass[
	%a4paper, % Use A4 paper size
	letterpaper, % Use US letter paper size
]{jdf}

\addbibresource{references.bib}

\author{William Luna}
\email{wluna6@gatech.edu}
\title{Assignment 1: Activity}

\begin{document}
%\lsstyle

\maketitle
%papers that were solving a problem or investigating a phenomenon
\section{\cite{nation1992vocabulary}}
\subsection{Full Reference}
%Introduce the paper with its full APA citation information and include a link to where the paper can be found
\fullcite{nation1992vocabulary}

\href{https://scholarspace.manoa.hawaii.edu/server/api/core/bitstreams/04d7edf5-be1c-4a1e-9c91-995135ac4120/content}{Link to Paper}

\subsection{Need}
%What is the need? What problem is trying to be solved or phenomenon trying to be investigated?
The title gives this away-the paper seeks to determine how much vocabulary is needed to be able to read un-simplified (ie aimed at native speakers) texts for pleasure? Pleasure being defined as the ability to read and comprehend the text without the need for effortful reading or active lookup of novel words.

\subsection{Method}
%What is method? What did they build to solve the problem? What did they do to investigate the phenomenon?
The paper analyzed three short novels aimed at young adults: \textit{The Pearl}, \textit{The Haunting}, and \textit{Alice in Wonderland}. Using \textit{A Teacher's Word Book of 30,000 Words}, an analysis is performed to determine how many of the most common N words in English a reader would have to know to know ninety-eight percent of the non-Proper nouns in these three texts.

\subsection{Audience}
%Who is the audience? Who are the participants or subjects?
This study did not have participants, being purely a textual analysis. The study does not imply a greater relevance for learners of English as a foreign language or native speakers learning to read as children, so relevance is presumed for both groups. The results of the study could inform how soon students engage with unsimplified novels during their language acquisition process.

\subsection{Results}
%What were the results? What did they find?
Despite the fact that each of the books only has about 2,000 unique words, these share limited overlap with the most common words in English. The analysis finds that a reader would have to know the most common 5,000 words in English to reach ninety-eight percent recognition.

\subsection{Critique}
%Critique the alignment between need, audience, method, and results.
%Are the results justified by the method? Does the method address the need? Is the audience fitting for the need?
%Overall, how strong is the alignment between need, audience, method, and results?
The conclusion drawn by the study appears sound. However, the premise of the study feels under-justified. The paper provides evidence that ninety-five percent comprehension is sufficient to "gain adequate comprehension and to guess unknown words from context." So the ninety-eight percent milestone appears contrived. However, the study's call for additional levels of graded readers to provide scaffolding for students in the 2,000-5,000 acquired word range stood to have a positive effect on the pedagogy at the time.

\section{\cite{hu_2000}}
\subsection{Full Reference}
%Introduce the paper with its full APA citation information and include a link to where the paper can be found
\fullcite{hu_2000}

\href{https://scholarspace.manoa.hawaii.edu/server/api/core/bitstreams/c5873d5c-23b5-41d1-99a5-fde539883ceb/content}{Link to Paper}
\subsection{Need}
%What is the need? What problem is trying to be solved or phenomenon trying to be investigated?
This was one of the first paper's to attempt to empirically assess the percentage of words in a text that had to be known by the reader for the content of that text to be deeply understood by the same reader. At the time (2000) few studies had rigorously prepared and administered the necessary study to a classroom of actual English language students.

\subsection{Method}
%What is method? What did they build to solve the problem? What did they do to investigate the phenomenon?
The researchers prepared a text written in simple English presumed to be easily comprehensible by all sixty-six of the ESL study participants. Then, the text was manipulated, with four different versions being created, one with no alterations, and three where the least common N percent of words are replaced with phonologically plausible nonsense words. N was set to 5, 10, and 20, resulting in texts where a reader is guaranteed to only understand a maximum of 95, 90, or 80 percent of the vocabulary, given that the remaining words don't exist. Study participants had to score sufficiently high on a vocabulary test to prove they would only struggle with nonsense words. Each participant was assigned to one of the four conditions, read the text, and took a comprehension test afterwards.

\subsection{Audience}
%Who is the audience? Who are the participants or subjects?
The audience are any learners of a foreign language, although the paper notes tenuous applicability outside of the college students to whom the study was administered. The findings may not apply to elementary school students, for example, or hobbyists who have less time to dedicate to the language each week.

\subsection{Results}
%What were the results? What did they find?
Despite variation in individual student performance, no group exposed to a version of the text including nonsense words was able to perform on par with the group with an unaltered text. There was a statistically significant decline in comprehension as more words were replaced with a nonsense equivalent. The study concludes that ninety-five percent comprehension of words in a text is therefore insufficient to comprehend a narrative text, supporting earlier research that ninety-eight percent is the necessary benchmark.

\subsection{Critique}
%Critique the alignment between need, audience, method, and results.
%Are the results justified by the method? Does the method address the need? Is the audience fitting for the need?
%Overall, how strong is the alignment between need, audience, method, and results?
I consider the Need, Method and Audience of the study to be aligned. Although I would have appreciated greater defense of why this experiment cannot be run on native speakers of the text's language, if nonsense words are being used. I am more critical of the results. They claim that this study provides "experimental support" for an earlier paper's claim that learners need ninety-eight percent coverage to be able to read for pleasure. However, there's no direct evidence to suggest this, only indirect evidence that fails to refute it. I am surprised it does not suggest the obvious follow-up study of administering the same experiment but with more conservative alterations to the text, say ninety-five, nine-eight, ninety-nine, and one-hundred percent original text, to have a more precise view of the ideal threshold.

What's even more interesting is this study \textit{does} refute research cited by the above paper from Laufer and Liu that ninety-five percent is an adequate threshold (\cite{nation1992vocabulary}). I would have appreciated recognition of this fact.

\section{\cite{wu2024indepth}}
\subsection{Full Reference}
%Introduce the paper with its full APA citation information and include a link to where the paper can be found
\fullcite{wu2024indepth}

\href{https://arxiv.org/pdf/2403.04963}{Link to Paper}

\subsection{Need}
%What is the need? What problem is trying to be solved or phenomenon trying to be investigated?
The paper calls out the importance of sentence simplification technology for any population with reading difficulties, but does not go into specifics (learning disabilities, foreign language learners, etc). Given the growing ubiquity of LLMs, assessing whether or not they can perform adequately at sentence simplification tasks is important for making written language more accessible. Therefore, the paper wants to determine if it's possible and if so how to best evaluate LLMs' performance at this task.

\subsection{Method}
%What is method? What did they build to solve the problem? What did they do to investigate the phenomenon?
The paper proposes a novel framework for human evaluation of LLM sentence simplification, which it applies alongside several automated metrics already established in the field. The goal of this method is two-fold: assess whether the novel framework is more effective than existing options, and assess how LLMs compare to neural network-based approaches.

\subsection{Audience}
%Who is the audience? Who are the participants or subjects?
Those who stand to benefit from the results are different from those engaged in the methodology. The human evaluators employed in annotating and rating the results of the LLM output do not have any presumed challenges with reading unsimplified text. The beneficiaries of the study, however, do.

\subsection{Results}
%What were the results? What did they find?
The study found that GPT-4 surpassed a SOTA (state-of-the-art) model on all aspects of performance, according to all available benchmarks, both human and automated, for evaluation.

The study also found that its proposed framework for human evaluation produced results with less variability between human reviewer. Despite this, it concludes that both these and automated benchmarks are too coarse to accurately distinguish between more minor gradations in model performance, where output may not contain a mistake but have leverage a sub-optimal phrasing or circumlocution.

\subsection{Critique}
%Critique the alignment between need, audience, method, and results.
%Are the results justified by the method? Does the method address the need? Is the audience fitting for the need?
%Overall, how strong is the alignment between need, audience, method, and results?
While thorough, I felt this lost some clarity of vision by attempting to simultaneously asses a novel evaluation framework and how existing frameworks benchmark GPT-4 against other sentence simplification systems. While the paper draws meaningful conclusions in both domains, it's easy to interpret each conclusion as frustrating the other, since one makes a statement about the limitation of evaluation frameworks, while the other leverages those same frameworks to make a statement about GPT-4's superior performance.

\section{\cite{Chang2015ImprovingRR}}
\subsection{Full Reference}
%Introduce the paper with its full APA citation information and include a link to where the paper can be found
\fullcite{Chang2015ImprovingRR}

\href{https://www.sciencedirect.com/science/article/pii/S0346251X15000846}{Link to Paper}

\subsection{Need}
%What is the need? What problem is trying to be solved or phenomenon trying to be investigated?
This paper is investigating whether vocabulary acquisition outcomes change (presumably for the better) by providing audio voice-overs to accompany text read during a silent reading exercise.

\subsection{Method}
%What is method? What did they build to solve the problem? What did they do to investigate the phenomenon?
Sixty-four EFL students where split into a control or experimental group, where the former engaged in silent reading for ninety minutes each week for twenty six weeks. The experimental group engaged in the same exercise, but while listening to audio that reads out the same text at a normal speech rate. Graded readers were used.

\subsection{Audience}
%Who is the audience? Who are the participants or subjects?
The EFL students are the subjects of the study and also the individuals who stand to benefit the most from the results of it. That being said, there may be relevant conclusions for populations that struggle to acquire native language due to intellectual disability.

\subsection{Results}
%What were the results? What did they find?
Pre- and post-tests revealed that not only did the experimental group have improved reading rates and comprehension at the end of the study period, but that these gains persisted three months after the reading sessions ended. This implies that extensive reading should always be accompanied by audio barring some good reason not to.

\subsection{Critique}
%Critique the alignment between need, audience, method, and results.
%Are the results justified by the method? Does the method address the need? Is the audience fitting for the need?
%Overall, how strong is the alignment between need, audience, method, and results?
Of all the papers I read this week, I felt this one had by far the most methodology. Significant sample size, diversity of texts (twenty graded readers), sufficiently longitudinal in both testing and administration, testing for both speed and comprehension, statistical clarity in analyzing the results. I suppose the study could have expanded the subject pool to explore the impact of student proficiency on the benefit of audio-assisted reading (all participants were beginners). 

\section{\cite{shortestpathrepetitionscheduling}}
\subsection{Full Reference}
%Introduce the paper with its full APA citation information and include a link to where the paper can be found
\fullcite{shortestpathrepetitionscheduling}

\href{https://dl.acm.org/doi/pdf/10.1145/3534678.3539081}{Link to Paper}

\subsection{Need}
%What is the need? What problem is trying to be solved or phenomenon trying to be investigated?

\subsection{Method}
%What is method? What did they build to solve the problem? What did they do to investigate the phenomenon?

\subsection{Audience}
%Who is the audience? Who are the participants or subjects?

\subsection{Results}
%What were the results? What did they find?

\subsection{Critique}
%Critique the alignment between need, audience, method, and results.
%Are the results justified by the method? Does the method address the need? Is the audience fitting for the need?
%Overall, how strong is the alignment between need, audience, method, and results?

\printbibliography{}

\end{document}

